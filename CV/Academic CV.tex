%%%%%%%%┬begL%%%%%%%%%%%%%%%%%%%%%%%%%%%%%%%%%
% Medium Length Graduate Curriculum Vitae
% LaTeX Template
%
% This template has been downloaded from:
% http://www.latextemplates.com
%
% Original author:
% Rensselaer Polytechnic Institute (http://www.rpi.edu/dept/arc/training/latex/resumes/)
%
% Important note:
% This template requires the res.cls file to be in the same directory as the
% .tex file. The res.cls file provides the resume style used for structuring the
% document.
%
%%%%%%%%%%%%%%%%%%%%%%%%%%%%%%%%%%%%%%%%%

%----------------------------------------------------------------------------------------
%	PACKAGES AND OTHER DOCUMENT CONFIGURATIONS
%----------------------------------------------------------------------------------------

\documentclass[margin]{res} % Use the res.cls style
\textheight 9.6in

\usepackage{helvet} % Default font is the helvetica postscript font
%\usepackage{newcent} % To change the default font to the new century schoolbook postscript font uncomment this line and comment the one above
\usepackage{hyperref}
\hypersetup{
	colorlinks=true,
	linkcolor=blue,
	filecolor=magenta,      
	urlcolor=blue,
}

\usepackage{multicol}

\setlength{\textwidth}{5.1in} % Text width of the document

\begin{document}

%----------------------------------------------------------------------------------------
%	NAME AND ADDRESS SECTION
%----------------------------------------------------------------------------------------

\moveleft.5\hoffset\centerline{\large\bf F. Javier Rodr\'iguez Rom\'an} % Your name at the top
 
\moveleft\hoffset\vbox{\hrule width\resumewidth height 1pt}\smallskip % Horizontal line after name; adjust line thickness by changing the '1pt'
 
\moveleft.5\hoffset\centerline{Av. Diagonal 696, office 408} % Your address
\moveleft.5\hoffset\centerline{08034, Barcelona, Spain}
\moveleft.5\hoffset\centerline{(+34) 622 925412}
\moveleft.5\hoffset\centerline{\href{mailto:fjavier.rodriguezroman@gmail.com}{fjavier.rodriguezroman@gmail.com}}
\moveleft.5\hoffset\centerline{\href{https://frarodri.github.io/}{frarodri.github.io}}
\moveleft.5\hoffset\centerline{Last updated: \today}
\moveleft.5\hoffset\centerline{Click \href{https://frarodri.github.io/CV/Academic\%20CV.pdf}{here} for the most up-to-date version}
	
%----------------------------------------------------------------------------------------

\begin{resume}

%----------------------------------------------------------------------------------------
%	EMPLOYMENT SECTION
%----------------------------------------------------------------------------------------
 
\section{Employment}
{\bf Universitat de Barcelona} \hfill September 2023-present  \\
{\sl Post-doctoral Researcher, Department of Economics} \\
{\bf University of Cagliari} \hfill June 2021-August 2023  \\
{\sl Post-doctoral Researcher, Department of Economics and Business} \\
{\bf The University of Edinburgh} \hfill October 2020-June 2021  \\
{\sl Stipendiary Fellow, School of Economics} 

%----------------------------------------------------------------------------------------
%	OTHER AFFILIATIONS
%----------------------------------------------------------------------------------------

\section{Other \\ affiliations}
{\bf Frankfurt Quantitative Macro Group} \hfill September 2023-present  \\
{\sl Junior Researcher} 

%----------------------------------------------------------------------------------------
%	EDUCATION SECTION
%----------------------------------------------------------------------------------------

\section{Education}
{\bf Universidad Carlos III de Madrid} \hfill June 2021  \\
{\sl Ph.D in Economics} \\
%{\bf Universidad Carlos III de Madrid} \hfill September 2016 \\
%{\sl Master in Economic Analysis} \\
{\bf Barcelona School of Economics} \hfill July 2013 \\
{\sl Master in Economics and Finance}  \\
{\bf Universidad de Costa Rica} \hfill August 2012 \\
{\sl B.A. in Economics}

\section{Academic \\ visits}
{\bf Centre for Macroeconomics} \hfill September-December 2022  \\
{\sl London School of Economics}

%\section{References}
%\begin{multicols}{2}
%	\href{http://http://www.alessiomoro.it/}{\bf{Alessio Moro}} \\
%	Department of Economics and Business, \\
%	University of Cagliari, \\
%	(+39) 0706753313 \\
%	\href{mailto:amoro@unica.it}{amoro@unica.it}
%	
%	\href{http://economics.uc3m.es/personal/andres-erosa/}{\bf{Andr\'es Erosa}}  \\
%	Department of Economics, \\
%	Universidad Carlos III de Madrid, \\
%	(+34) 916249600 \\
%	\href{mailto:aerosa@eco.uc3m.es}{aerosa@eco.uc3m.es}
%	
%	\columnbreak
%	
%	\href{http://www.eco.uc3m.es/~mkredler/}{\bf{Matthias Kredler}} (Ph.D advisor)\\
%	Department of Economics, \\
%	Universidad Carlos III de Madrid, \\
%	(+34) 916249312 \\
%	\href{mailto:matthias.kredler@uc3m.es}{matthias.kredler@uc3m.es}
%	
%	\href{http://economics.uc3m.es/personal/luisa-fuster/}{\bf{Luisa Fuster}} \\
%	Department of Economics, \\
%	Universidad Carlos III de Madrid, \\
%	(+34) 916249331 \\
%	\href{mailto:lfuster@eco.uc3m.es}{lfuster@eco.uc3m.es}
%\end{multicols}

\section{Fields}

Macroeconomics, Labour, Demography and Family Economics.

\section{Working \\ papers}
{\bf The Sex Ratio, Marriage, and Bargaining: A Look at China}  \\
\textit{Revision Requested, Review of Economic Dynamics} \\
---Nominated for best early-career paper at the 1st International Workshop on the Chinese Development Model (Barcelona, Spain, July 2022). 

%In this paper I develop a model of marriage, bargaining and time allocation to assess the quantitative importance of changes in the sex ratios on paid work, housework, leisure and assortative mating. I then calibrate the model with Chinese data, as the country has been experiencing a surge in boy births relative to girls' since the 1980s. I find that changes in the sex ratio explain around half of the changes in married women paid work and leisure time between 1990 and 2010. Moreover, I find that the effect of the sex ratio operated mainly through bargaining within the household, and very marginally via marital sorting. Moreover, low and medium skilled women are the ones that have seen the largest improvement in their bargaining position.

{\bf Cash Transfers and Fertility: From Short to Long Run} (with Lidia Cruces)
---Best paper award at the 2023 Spring Meeting of Young Economists (Turin, Italy, September 2023). 

%Many developed countries are at risk of experiencing population decline due to low fertility rates, with potential negative economic effects. As a response, governments are deploying family policies to increase the number of children. In this paper, we propose a dynamic life-cycle model of fertility and female labour force participation to assess their effectiveness. We use the short-run fertility effects of a cash transfer policy from Spain to calibrate its parameters. Using the calibrated model, we find that the effects in the long run are half as large as in the short run. This is driven by differences in the responses of younger and older women at the time of implementation. The latter must react shortly after, as they cannot delay fertility much longer. The former anticipate their first birth. This generates additional births in the short run. We also study the effects of an alternative policy consisting of childcare subsidisation, and explore how the coexistence of temporary and permanent contracts in Spain, which have different earnings profiles, affects fertility and interacts with cash transfers, by raising the costs of career interruptions in crucial child-bearing years.

%{\bf Quantifying the Impact of Childcare Subsidies on Social Security} (with Lidia Cruces) \\
%Female labour force participation and fertility levels directly impact social security, especially when it relies on a pay-as-you-go scheme. In this paper, we quantify the impact of childcare subsidisation policies on a PAYG social security system. We build an overlapping generations model in which women decide how many children to have, the allocation of childcare time among different alternatives and their labour force participation along their life cycle. We calibrate the model to Spanish data and use it to experiment with different childcare subsidisation policies. We find that childcare subsidies increase mother's labour force participation and fertility minimally. Therefore, they have a negative effect on the present value of social security budget balance. 

{\bf A Theory of Structural Change, Home Production and Leisure} (with Fenicia Cossu, Alessio Moro and Silvio Tunis) 

\section{Work in \\ progress}
{\bf Slums and Urbanisation Without Structural Transformation} (with Alessio Moro) \\
---Funded with a Small Research Grant from the Structural Transformation and Economic Growth (STEG) programme (£25000). \\
{\bf Family Policies and Social Security} (with Lidia Cruces)

\section{Presentations}

{\bf 2023:} Universidad Alberto Hurtado (online), CUNEF, Goethe University, PUC Chile, Monash, SAET Conference* (Paris, France), Spring Meeting of Young Economists (Turin, Italy). {\bf 2022:} Workshop on Structural Transformation and Macroeconomic Dynamics (Cagliari, Italy), International Workshop on the Chinese Development Model (Barcelona, Spain), STEG Theme 2 Workshop* (online), Symposium of the Spanish Economic Association* (Valencia, Spain). {\bf 2021:} Macroeconomics Working Group EUI (online), European Society for Population Economics (online), Brazilian Meeting in Family and Gender Economics (online), CRENoS Workshop (Asinara, Italy), Symposium of the Spanish Economic Association (Barcelona, Spain). {\bf 2020:} Macro-Development-Trade-Environment Reading Group Workshop at Iowa State University (online). {\bf 2019:} Workshop on Dynamic Macroeconomics (Vigo, Spain), University of Mannheim, Symposium of the Spanish Economic Association (Alicante, Spain).

* Denotes presentation by co-author.

%----------------------------------------------------------------------------------------
%	COMPUTER SKILLS SECTION
%----------------------------------------------------------------------------------------

\section{Scholarships, \\ grants and \\ awards} 

STEG Small Research Grant (£25000) \hfill 2023 \\
SAEe PhD grant \hfill 2019\\
FPI Doctoral Scholarship, Spanish Ministry of Science \hfill 2016-2020 \\
UC3M Economics Department scholarship \hfill 2014-2016 \\
Bolsa Inci, IMPA (Brazil) \hfill 2012 \\
Beca 10 Universidad de Costa Rica \hfill 2007

%----------------------------------------------------------------------------------------
%	TEACHING EXPERIENCE SECTION
%----------------------------------------------------------------------------------------

\section{Teaching experience}

{\bf The University of Edinburgh:} Course Organiser for International Economics (Autumn 2020). Tutor for Economics 1 (Autumn 2020), Economics 2 (Spring 2021), Introductory Financial Economics (Spring 2021) and Topics in Macroeconomics (Spring 2021).
{\bf Universidad Carlos III de Madrid:} Teaching assistant for Industrial Organization I: Theory and Regulation (graduate, Autumn 2018), Quantitative Microeconomics (Autumn 2017 and 2018), Principles of Economics (Spring 2017, Autumn 2016 and 2018), Dynamic Macroeconomics (Spring 2016, 2017 and 2018), International Trade (Autumn 2017)

%{\bf As TA at Universidad Carlos III de Madrid} \\
%Industrial Organization I: Theory and Regulation (graduate) \hfill Fall 2018 \\
%Instructor: Matilde Machado \\
%Quantitative Microeconomics (undergraduate) \hfill Fall 2017 and 2018 \\
%Instructors: Iliana Reggio and Ricardo Mora \\
%Principles of Economics (undergraduate) \hfill Spring 2017, Fall 2016 and 2018 \\
%Instructors: Antonio Romero, Nicolas Motz and Asier Mariscal \\
%Dynamic Macroeconomics (undergraduate) \hfill Spring 2016, 2017 and 2018 \\
%Instructor: Luisa Fuster \\
%International Trade (undergraduate) \hfill Fall 2017 \\ 
%Instructor: Miguel Marinas 

%{\bf As instructor at Universidad de Costa Rica} 
%
%Quantitative Methods in Economics (undergraduate)  \hfill Spring 2014

\section{Professional \\ service}
\textbf{Refereeing}: Journal of Population Economics, Econom\'ia, Review of Economics of the Household, Macroeconomic Dynamics. \\
-Member of the board of the European Association of Young Economists, 2023-

\section{Software and \\ programming}
Stata, Matlab, SPSS, Eviews, Python. \\

\section{Other}
Languages: Spanish (native), English (fluent), Italian (fluent). \\
Citizenship: Costa Rican.

%\section{REFERENCES}
%\mbox{} \\
%{\bf Eduardo Pedreira Collazo}\\
%Senior Economist, "la Caixa", Barcelona, Spain\\
%eduardo.pedreira@lacaixa.com 

%{\bf Albrecht Glitz}\\
%Universitat Pompeu Fabra\\
%albrecht.glitz@upf.edu

%{\bf Juan Robalino Herrera}\\
%Universidad de Costa Rica, CATIE\\
%robalino@catie.ac.cr
%----------------------------------------------------------------------------------------

\end{resume}
\end{document}