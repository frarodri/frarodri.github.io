%%%%%%%%┬begL%%%%%%%%%%%%%%%%%%%%%%%%%%%%%%%%%
% Medium Length Graduate Curriculum Vitae
% LaTeX Template
%
% This template has been downloaded from:
% http://www.latextemplates.com
%
% Original author:
% Rensselaer Polytechnic Institute (http://www.rpi.edu/dept/arc/training/latex/resumes/)
%
% Important note:
% This template requires the res.cls file to be in the same directory as the
% .tex file. The res.cls file provides the resume style used for structuring the
% document.
%
%%%%%%%%%%%%%%%%%%%%%%%%%%%%%%%%%%%%%%%%%

%----------------------------------------------------------------------------------------
%	PACKAGES AND OTHER DOCUMENT CONFIGURATIONS
%----------------------------------------------------------------------------------------

\documentclass[margin]{res} % Use the res.cls style
\textheight 9.6in

\usepackage{helvet} % Default font is the helvetica postscript font
%\usepackage{newcent} % To change the default font to the new century schoolbook postscript font uncomment this line and comment the one above
\usepackage{hyperref}
\hypersetup{
	colorlinks=true,
	linkcolor=blue,
	filecolor=magenta,      
	urlcolor=blue,
}

\usepackage{multicol}

\setlength{\textwidth}{5.1in} % Text width of the document

\begin{document}

%----------------------------------------------------------------------------------------
%	NAME AND ADDRESS SECTION
%----------------------------------------------------------------------------------------

\moveleft.5\hoffset\centerline{\large\bf Francisco Javier Rodr\'iguez Rom\'an} % Your name at the top
 
\moveleft\hoffset\vbox{\hrule width\resumewidth height 1pt}\smallskip % Horizontal line after name; adjust line thickness by changing the '1pt'
 
\moveleft.5\hoffset\centerline{Vico II Vincenzo Sulis, 5 (cit. Garau)} % Your address
\moveleft.5\hoffset\centerline{09124, Cagliari, Italy}
\moveleft.5\hoffset\centerline{(+39) 333 3560359}
\moveleft.5\hoffset\centerline{\href{mailto:fjavier.rodriguezroman@gmail.com}{fjavier.rodriguezroman@gmail.com}}
\moveleft.5\hoffset\centerline{\href{https://frarodri.github.io/}{frarodri.github.io}}
\moveleft.5\hoffset\centerline{Last updated: \today}

%----------------------------------------------------------------------------------------

\begin{resume}

%----------------------------------------------------------------------------------------
%	EMPLOYMENT SECTION
%----------------------------------------------------------------------------------------
 
\section{Employment}
{\bf University of Cagliari} \hfill June 2021-present  \\
{\sl Post-doctoral Researcher, Department of Economics and Business} \\
{\bf University of Edinburgh} \hfill October 2020-June 2021  \\
{\sl Stipendiary Fellow, School of Economics} 

%----------------------------------------------------------------------------------------
%	EDUCATION SECTION
%----------------------------------------------------------------------------------------

\section{Education}
{\bf Universidad Carlos III de Madrid} \hfill June 2021  \\
{\sl Ph.D in Economics} \\
{\bf Universidad Carlos III de Madrid} \hfill September 2016 \\
{\sl Master in Economic Analysis} \\
{\bf Barcelona Graduate School of Economics} \hfill July 2013 \\
{\sl Master in Economics and Finance}  \\
{\bf Universidad de Costa Rica} \hfill August 2012 \\
{\sl B.A. in Economics}

\section{References}
\begin{multicols}{2}
	
	\href{http://http://www.alessiomoro.it/}{\bf{Alessio Moro}} \\
	Department of Economics and Business, \\
	University of Cagliari, \\
	(+39) 070/675-3313 \\
	\href{mailto:amoro@unica.it}{amoro@unica.it}
	
	\href{http://economics.uc3m.es/personal/andres-erosa/}{\bf{Andr\'es Erosa}}  \\
	Department of Economics, \\
	Universidad Carlos III de Madrid, \\
	(+34) 916249600 \\
	\href{mailto:aerosa@eco.uc3m.es}{aerosa@eco.uc3m.es}
	
	\columnbreak
	
	\href{http://www.eco.uc3m.es/~mkredler/}{\bf{Matthias Kredler}} (Ph.D advisor)\\
	Department of Economics, \\
	Universidad Carlos III de Madrid, \\
	(+34) 916249312 \\
	\href{mailto:matthias.kredler@uc3m.es}{matthias.kredler@uc3m.es}
	
	\href{http://economics.uc3m.es/personal/luisa-fuster/}{\bf{Luisa Fuster}} \\
	Department of Economics, \\
	Universidad Carlos III de Madrid, \\
	(+34) 916249331 \\
	\href{mailto:lfuster@eco.uc3m.es}{lfuster@eco.uc3m.es}
	
\end{multicols}

\section{Fields}

Macroeconomics, Family Economics, Demographic Economics, Labor Economics.

\section{Working papers}
{\bf Quantifying the Impact of Childcare Subsidies on Social Security} (joint with Lidia Cruces) \\
Female labour force participation and fertility levels directly impact social security, especially when it relies on a pay-as-you-go scheme. In this paper, we quantify the impact of childcare subsidisation policies on a PAYG social security system. We build an overlapping generations model in which women decide how many children to have, the allocation of childcare time among different alternatives and their labour force participation along their life cycle. We calibrate the model to Spanish data and use it to experiment with different childcare subsidisation policies. We find that childcare subsidies increase mother's labour force participation and fertility minimally. Therefore, they have a negative effect on the present value of social security budget balance. 

{\bf The Sex Ratio, Marriage, and Bargaining: a Look at China} \\
In this paper I develop a model of marriage, bargaining and time allocation to assess the quantitative importance of changes in the sex ratios on paid work, housework, leisure and assortative mating. I then calibrate the model with Chinese data, as the country has been experiencing a surge in boy births relative to girls' since the 1980s. I find that changes in the sex ratio explain around half of the changes in married women paid work and leisure time between 1990 and 2010. Moreover, I find that the effect of the sex ratio operated mainly through bargaining within the household, and very marginally via marital sorting. Moreover, low and medium skilled women are the ones that have seen the largest improvement in their bargaining position.

\section{Work in progress}

{\bf Child subsidies, female labour force participation
and fertility: from short to long-run} (with Lidia Cruces)
\\
{\bf A Theory of Structural Change, Home Production and Leisure} (with Fenicia Cossu, Alessio Moro and Silvio Tunis)

\section{Presentations}

{\bf 2021:} Macroeconomics Working Group at the European University Institute (online), 34th Annual Conference of the European Society for Population Economics (online), 2nd Brazilian Meeting in Family and Gender Economics (online), CRENoS Workshop (Asinara, Italy), 46th Symposium of the Spanish Economic Association (Barcelona, Spain).

{\bf 2020:} Virtual Macro-Development-Trade-Environment Reading Group Workshop at Iowa State University (online).

{\bf 2019}: XXIV Workshop on Dynamic Macroeconomics (Vigo, Spain), Macro Seminar at the University of Mannheim, 44th Symposium of the Spanish Economic Association (Alicante, Spain).

%----------------------------------------------------------------------------------------
%	COMPUTER SKILLS SECTION
%----------------------------------------------------------------------------------------

\section{Scholarships \\ and awards} 

SAEe PhD grant \hfill 2019\\
FPI Doctoral Scholarship (Spanish Ministry of Science) \hfill 2016-2020 \\
UC3M Economics Department scholarship \hfill 2014-2016 \\
Honor Roll Universidad de Costa Rica (Licenciatura) \hfill 2012 \\ 
Bolsa Inci (IMPA, Brazil) \hfill 2012 \\
Beca 10 Universidad de Costa Rica \hfill 2007

%----------------------------------------------------------------------------------------
%	TEACHING EXPERIENCE SECTION
%----------------------------------------------------------------------------------------

\section{Teaching experience}

{\bf As Course Organizer at The University of Edinburgh} \\
International Economics \hfill Fall 2020

{\bf As TA at Universidad Carlos III de Madrid} \\
Industrial Organization I: Theory and Regulation (graduate) \hfill Fall 2018 \\
Instructor: Matilde Machado \\
Quantitative Microeconomics (undergraduate) \hfill Fall 2017 and 2018 \\
Instructors: Iliana Reggio and Ricardo Mora \\
Principles of Economics (undergraduate) \hfill Spring 2017, Fall 2016 and 2018 \\
Instructors: Antonio Romero, Nicolas Motz and Asier Mariscal \\
Dynamic Macroeconomics (undergraduate) \hfill Spring 2016, 2017 and 2018 \\
Instructor: Luisa Fuster \\
International Trade (undergraduate) \hfill Fall 2017 \\ 
Instructor: Miguel Marinas 

%{\bf As instructor at Universidad de Costa Rica} 
%
%Quantitative Methods in Economics (undergraduate)  \hfill Spring 2014

\section{Software and \\ programming}
Stata, Matlab, SPSS, Eviews, Python. \\

\section{Other}
Refereeing: Journal of Population Economics, Econom\'ia (Journal of LACEA) \\
Languages: Spanish (native), English (fluent), Italian (basic). \\
Citizenship: Costa Rican.


%\section{REFERENCES}
%\mbox{} \\
%{\bf Eduardo Pedreira Collazo}\\
%Senior Economist, "la Caixa", Barcelona, Spain\\
%eduardo.pedreira@lacaixa.com 

%{\bf Albrecht Glitz}\\
%Universitat Pompeu Fabra\\
%albrecht.glitz@upf.edu

%{\bf Juan Robalino Herrera}\\
%Universidad de Costa Rica, CATIE\\
%robalino@catie.ac.cr
%----------------------------------------------------------------------------------------

\end{resume}
\end{document}